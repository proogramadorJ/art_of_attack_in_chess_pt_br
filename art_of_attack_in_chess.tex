	\documentclass[12pt, openright]{book}
	\usepackage{a4wide}
	\usepackage{array}
	\newcolumntype{C}{>{\centering\arraybackslash}X}
	\renewcommand{\arraystretch}{1.2}
	\usepackage{booktabs}
	\usepackage{tabularx}
	\usepackage{caption}
	\usepackage{array}
	\usepackage{float}
	\usepackage{natbib}
	\usepackage{graphicx}
	\usepackage{blindtext}
	\usepackage[pagestyles]{titlesec} 
	
	
	
	\usepackage{tocloft}
	
	
	% ===== Define abstract environment =====
	\newcommand{\prefacename}{Prefacio por Hohn Nunn}
	\newenvironment{preface}{
		\vspace*{\stretch{2}}
		{\noindent \bfseries \Huge \prefacename}
		\begin{center}
			% \phantomsection \addcontentsline{toc}{chapter}{\prefacename} % enable this if you want to put the preface in the table of contents
			\thispagestyle{plain}
		\end{center}%
	}
	{\vspace*{\stretch{5}}}
	
	\begin{document}
		\title{\textbf{Arte do ataque no xadrez}}
		\author{Vladimir Vukovic}
		\date{2008}
		\maketitle
		\pagestyle{empty}
		
	
		
		\shipout\null
		
		\frontmatter
		\pagenumbering{Roman} 
		
		\begin{preface}
			Atacar o rei inimigo é uma das partes mais emocionantes do xadrez, mas também uma das mais difíceis de jogar com precisão. Todo jogador de xadrez teve a experiência de ver um ataque de aparência promissora virar poeira, após o contra-ataque inimigo varrer tudo ao seu redor. O excelente livro \textit{Arte do ataque no xadrez} de Vukovic é uma tentativa instigante de explicar porque alguns ataques resultam em sucesso enquanto outros falham em atingir seu objetivo.
		\end{preface}
		
		\clearpage
		
		\tableofcontents
		
	\clearpage
		
		\assignpagestyle{\chapter}{plain}
		\section{Introdução}
		 TO DO
		
	%	\clearpage
	%	\listoffigures
		
	%	\clearpage
	%	\listoftables
		
		
		
	%	\mainmatter
		
	%	\Blinddocument
		
	\end{document}